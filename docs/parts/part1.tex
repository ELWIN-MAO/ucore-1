\part{系统启动}

\section*{\HandRight\hspace{0.5em}实验目的}
\begin{minipage}[t]{0.8\textwidth}
\begin{heiti}
\begin{itemize}
\renewcommand\labelitemi{\FiveStarShadow}
    \item \begin{large} 了解内核加载的过程 \end{large}
    \item \begin{large} 掌握内核调试的方法 \end{large}
    \item \begin{large} 熟悉分段的内存管理以及设备管理的概念 \end{large}
    \item \begin{large} 学习中断触发机制以及处理流程 \end{large}
\end{itemize}
\end{heiti}
\end{minipage}

\chapter{系统加载和启动}

操作系统本身也是一种软件,
也需要通过某种环境和机制来加载和运行它。
在这里,我们首先通过一个更加简单的软件 \cndash{\it{bootloader}} 来完成这样的工作。
\ucore 里面,我们提供了一个非常小的 bootloader,
整个程序编译后的大小不超过一个扇区(512 byte),
这样才能放到硬盘的主引导扇区。

这部分所有章节,围绕着 lab1,
主要介绍了\ucore 是如何加载和运行、如何处理和分发中断以及如何完成硬件自我调试的。
过程中,我们会逐渐熟悉 x86 硬件平台以及各种调试手段。
为后面完成更复杂的系统模块积累经验。

\section{了解 bootloader}

lab1 中包含了三部分与 bootloader 有关的小例子:
\begin{itemize}
    \item %
        {\bf{project1:}}
        bootloader 完成从 x86 实模式到保护模式的切换;
    \item %
        {\bf{project2:}}
        bootloader 实现简单的磁盘驱动,
        能够从磁盘读取和加载 ELF 可执行格式的程序;
    \item %
        {\bf{project3:}}
        bootloader 将 \ucore 内核从磁盘 ucore.img 中读取,
        然后加载到内存,并最终将 CPU 交给 \ucore。 
\end{itemize}

下图描述了三个 project 之间的关系,
`$\rightarrow$' 描述两个 project 之间的依赖关系
\footnote{\ucore 是按照 project 的顺序逐步实现,
直到最后完成一个完整的内核系统。
每个 project 都在前一个 project 的基础上做最小修改,
我们可以通过 diff 直接观察到这种代码上的变化。}:

\begin{footnotesize}
\begin{figure}[h]
\centering
\begin{tikzpicture}[x=0.05\textwidth,y=1em,auto]
    \tikzstyle{every node}=[fill=none,rectangle,draw,node distance=0pt,inner sep=3pt,outer sep=3pt]

    \node (proj1) {\makebox[0.15\textwidth]{project 1}};
    \draw let \p1=(proj1.east) in %
        [->] (\p1) -- +(20pt,0) node[anchor=west] (proj2) {\makebox[0.15\textwidth]{project 2}};
    \draw let \p1=(proj2.east) in %
        [->] (\p1) -- +(20pt,0) node[anchor=west] (proj3) {\makebox[0.15\textwidth]{project 3}};
    \draw let \p1=(proj3.east) in %
        [->] (\p1) -- +(10pt,0) node [draw=none,anchor=west] (A) {...};

\end{tikzpicture}
\end{figure}
\end{footnotesize}

\clearpage

project 1 是最小的 bootloader 例子,整个目录十分简单,结构如下:
\begin{figure}[h]
    \centering
    \scalebox{0.8}{
        \begin{minipage}{\textwidth}
            \setlength{\baselineskip}{1.3em}
            \newcommand{\minibox}[1]{%
            \begin{minipage}[t]{0.7\textwidth}\setlength{\baselineskip}{1.2em}%
                {#1}\end{minipage}}
            \dirtree{%
                .1 \bf{proj1/}.
                .2 \bf{boot}.
                .3 asm.h.
                .3 bootasm.S.
                .3 bootmain.c.
                .2 \bf{libs}.
                .3 types.h.
                .3 x86.h.
                .2 \bf{tools}.
                .3 function.mk.
                .3 gdbinit.
                .3 sign.c.
            }
        \end{minipage}
    }
\end{figure}

其中一些比较重要的文件说明如下:
\begin{footnotesize}
\begin{itemize}
    \item %
        {\bf{bootasm.S:}}
        定义并实现了 bootloader 的入口函数 start。
        此函数进行了一系列初始化操作,
        然后完成了从实模式到保护模式的转换。
        在设置好函数栈以后,通过跳转指令,
        跳转到 bootmain.c 中的 bootmain 函数,完成从汇编语言到高级语言的跳转。
\end{itemize}
\end{footnotesize}

\chapter{中断和异常处理}

\chapter{硬件调试}
